%Abstract will go here.  Make sure it remains within the 350 word limit.
%\par To get a new paragraph
This document contains discussions on completed and ongoing projects that have developed over the past few years 
while working on the ATLAS detector located at CERN in Geneva, Switzerland. The first discussion will be on my 
qualification task for the ATLAS collaboration which is on the development and future plans of the DL1d tagger
that is currently used as a baseline tagger for run 4 of the ATLAS detector. After, the discussion will 
transition to an analysis that applies a novel anomaly detection technique which uses a neural network 
architecture called the autoencoder. This neural network is then trained on 1\% randomly selected events of 
run 2 data from the ATLAS detector. Once the model and anomalous regions are defined, the model is used to find 
phase spaces where events that contain physics beyond the standard model may occur. Statistical analysis is 
then applied to these phase spaces in order to find signatures of new physics. No significant signatures are 
found. Lastly, I will discuss an ongoing search for a new massive scalar X decaying into a new light scalar Y
and the standard model Higgs boson H through the process X→YH→bbbb in the boosted topology. 
\par 

