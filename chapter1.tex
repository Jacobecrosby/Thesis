
\begingroup
\clearpage% Manually insert \clearpage
\let\clearpage\relax% Remove \clearpage functionality
\vspace*{-16pt}% Insert needed vertical retraction
\chapter[INTRODUCTION]{INTRODUCTION}\label{chap:intro}
\endgroup

As humans we are natural explorers, there are no challenges too daunting, no sought truth to be overwhelming. 
The thirst for answers about ourselves and our existence has driven the minds of men for centuries. This drive
has started to uncover the possibility of almost limitless potential. Our ancestors have gazed into the face 
of the night sky for thousands of years, asking, dreaming, and fantasizing of stories from the beginning. As 
we've evolved and developed new tools and the footprints of society looming in the horizon, the same sky 
still gazed down, fueling our drive to muster forward. The pureness of our curiosity passed through generations
started bearing fruit as new technologies were being developed. Logical reasoning and objective views of the 
the world in front of us stemmed hypotheses and models for its description and developed what we know today as 
science. Knowledge advanced so far that one day we split the atom, introducing a new era of science and physics, one of 
underlying anxiousness and fear but also excitement. These efforts towards understanding stretched and weaned its way 
into every facet of society giving us technology and access that previous civilizations would deem godlike. 
Here we find ourselves, delving even deeper into the field of physics. Searching for answers, patterns, 
and possibilities on every scale we can detect. The scale in which the following studies are performed is in 
the scale of high energy. 
\par

All the energy in the universe was produced in a single event called the Big Bang. In the moments
after the Big Bang occurring, all of this energy was compacted in an extremely small volume as it rapidly 
expanded outwards. This environment created conditions that gave particles an enormous amount of energy, causing their 
relationships to be much different than what we see today. The parameters of this environment no longer exists anywhere
in the universe except in particle accelerators here on Earth. These accelerators allow physicists to peer into this
hostile environment in order to gain a better understanding of the forces of nature and how they started. These accelerators
can have a range of energies. Lower energies would be considered Nuclear Physics which is the study of the nuclei and 
the nucleus's atoms. The name for the atom was given by John Dalton in the 19\textsuperscript{th} century after the Greek word \textit{atomos}
or "indivisible" \cite{Grossman}. Today we now know that an atom is not "indivisible" but is actually composed of smaller constituents called quarks.
This scale can only be obtained with particle accelerators at higher energies. Thus, the study of High Energy Physics (\gls{hep}), or Particle
Physics, is the study of the fundamental particles and the forces that connect them in the universe.
\par

The leading institution of Particle Physics is the Conseil Européen pour la Recherche Nucléaire or the European Organization for Nuclear Research 
(\gls{cern}) located in Geneva, Switzerland. This institution is the world's largest particle accelerator with the Large Hadron Collider (LHC) being a  27 kilometer
ring consisting of superconducting magnets along with a number of particle accelerating contraptions.
During the 1950s and 1960s particle accelerators were designed for much higher energies than accelerators at that time. Within this new energy threshold, 
the renaissance of Particle Physics began. The majority of particles that exist aren't stable and are only produced via highly 
energetic events, so a perplexing amount of particles were observed for the first time in scattering experiments. These two decades
were referred to as the "particle zoo". This term was no longer used in the early 1970s after the formulation of 
the Standard Model. This model is the foundation of Particle Physics and explains that these particles were a composition of a few much smaller 
and fundamental particles. 
\\

\section{Outline}

In chapter 2 we will discuss the Standard Model in much greater detail. Starting with its history
and its obvious motivation. We will delve deeper into the theory, showing its beauty and discussing the 
importance of its creation. This will lead us into the physical signatures that the Standard Model explains
and also further predicts which will lead us into the exciting discoveries it has made decades later.
\par
Chapter 3 follows up this discussion with explanations of tools developed in order to detect such signatures.
This leads us into the birth of the Large Hadron Collider (LHC) and one its largest detectors, ATLAS.
The inner workings of this detector will be explained along with
how they paint a beautiful picture of the chaos that occurs inside. Not only is 
the current status of the ATLAS detector discussed but also its upgrade which is scheduled in 2029. 
\par
After the workings of the ATLAS detector is well established, chapter 4 will introduce object reconstruction,
identification, and event simulation. From here we'll see how energy deposits within \gls{atlas} leads to low level triggers and up to 
high level kinematic reconstruction. This will then lead us into the next section.
\par
Once a full, reconstructed picture of the events that happen inside the ATLAS detector is well understood, chapter 5 will
explain the complex and state of the art software tools developed and high level in order to create order from the chaos. 
This will transition into my work for my qualification task that helped create one of these tools using machine learning 
for the coming upgrade in run 4 for the ATLAS detector. 
\par
Now that the full picture of energy deposition in the ATLAS detector leading to high level object reconstruction is well understood,
a discussion on finding Beyond the Standard Model (\gls{bsm}) signatures can begin. Here, in chapter 6, a new and innovative 
technique is described that was created to find such signatures. This technique uses an agnostic and unsupervised machine
learning approach to find anomalies within data of the ATLAS detector. This chapter covers from start to finish an entire
analysis along with its findings.
\par
Chapter 7 will finish the thesis with the application of this innovative approach into a non-agnostic \gls{bsm} search.
This analysis is still preliminary and its plots should not be taken as final results. However, it generates interesting 
discussion on adapting this anomaly detection technique. This analysis will be continued by another student. 
\par
Finally, chapter 8 will have the closing discussing and conclusions of this dissertation. Summing up three years 
of hard work and marking the beginning of another chapter in life. 