\begingroup
\clearpage% Manually insert \clearpage
\let\clearpage\relax% Remove \clearpage functionality
\vspace*{-16pt}% Insert needed vertical retraction
\chapter[CONCLUSION]{CONCLUSION}
\endgroup

The \gls{sm} of particle physics has proven to be one of the most successful theories that has ever been developed, standing firm through decades of research while being 
solidified more through the discovery of the Higgs boson. Even though its success is unprecedented, there are gaps that need to be filled in order to describe phenomena 
such as gravity and dark matter. As the field of High Energy Physics grows to help explain and fill these gaps, new detectors, techniques and tools are needed to be 
developed. This dissertation presents two studies that help this ever needed growth, along with a proposed non-generic anomaly detection search that is in its infant stage.
One presents preliminary studies of particle identification machine learning tools using 
simulated data of the \gls{atlas} detector with its upgrade that is being implemented in 2029 for the evolution of the \gls{hllhc}. The second study shows a full analysis 
that was the first of its kind using unsupervised machine learning for anomaly detection. A non-generic anomaly detection analysis with very preliminary plots is also discussed. 
\par
The DL1d tagger developed for the \gls{atlas} Upgrade for the \gls{hllhc} shows promising progress. This is known to be preliminary due to the fact that the training statistics,
which is crucial for its effectiveness, has a magnitude less than the current \gls{atlas} standard tagger for Run 2. Regardless of this fact, the trained DL1d tagger for Upgrade
shows to be as effective at higher working points while also allowing for the tagging of particles at $|\textrm{η}| \ge$ 2.4 due to the new implemented \gls{atlas} geometry. 
The DL1d tagger is in the works of being replaced by a tagger based on a graph neural network architecture and is planned to the standard used tagger once the Upgrade is complete 
in 2029. Therefore, as of right now, there is no plan to retrain the DL1d tagger for Upgrade and the one that is trained in this dissertation will be used as the industry baseline. 
\par
The final analysis shows a novel anomaly detection technique using unsupervised machine learning for anomaly detection training on 1\% of Run 2 data. The deep-learning architecture
of an \gls{ae} is used with its reconstruction loss defining the anomaly score. From here, anomaly regions are defined by using the anomaly score as cut on \gls{sm} events.
Bump hunting strategies are discussed and implemented on nine different di-object invariant mass distributions in order to find any deviations within these anomaly regions 
that may hint at signatures of new physics. The largest deviation found is within the \mjmu distribution in the mass range of 4.6-4.8 TeV. This deviation resulted in a 
local significance of 3.9$\sigma$ and a global significance of 2.2$\sigma$. Frequentist limits were taken for Gaussian shape signals with means at a ranges defined by 
benchmark \gls{bsm} models. 
\par
The beginning of a non-generic anomaly detection search using the strategy discussed in this analysis for a specific \gls{bsm} model of a heavy scalar boson is briefly mentioned. 
The plots shown are very preliminary. A fellow graduate student is expected to continue this analysis in hopes it may discover hints of new physics. 
